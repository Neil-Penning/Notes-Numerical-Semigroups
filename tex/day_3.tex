\chapter{Faces of Polyhedra}

\begin{boxNote}{}
    Here are some properties to test:\\
    \begin{boxProblem}{}
        For a polyhedra \( P \), \( \emptyset \) is a face of \( P \)
        \tcblower
    \end{boxProblem}
    \begin{boxProblem}{}
        For a polyhedra \( P \), \( P \) itself is a face of \( P \)
        \tcblower
    \end{boxProblem}
    \begin{boxProblem}{}
        For any Polytope \( P \), face \( F \) is the convex hull of vertices of \( P \) it contains.
        \begin{boxNote}{V-Description}
            This is called the V-Description.
        \end{boxNote}
    \end{boxProblem}
    \begin{boxProblem}{}
        Any face \( F \) is the intersection of the faces of \( P \) it contains.
        \begin{boxNote}{H-Description}
            This is called the H-Description.
        \end{boxNote}
    \end{boxProblem}
\end{boxNote}

\section{Face Structure of Polyhedra and Polytopes}
\begin{boxDefinition}{The Face Lattice of a Polytope}
    This is a poset via containment.
    \begin{boxTODO}{}
        Draw or copy from Thomas' book.
    \end{boxTODO}
\end{boxDefinition}
\begin{boxTheorem}{}
    A \( k \)-dimensional simplex has a face lattice that looks like a \( (k+1) \)-dimensional cube.
\end{boxTheorem}
\begin{boxTheorem}{}
    A \( k \) dimensional simplex has a face lattice isomorphic to the poset of the powerset on \( (k+1) \)-elements.
    \begin{boxTheorem}{}
        
    \end{boxTheorem}
\end{boxTheorem}
\begin{boxNote}{}
    The degree of \( \emptyset \) is -1, by convension.
\end{boxNote}
\begin{boxNote}{}
    A face lattice will always have "nice" rows, via dimension.
\end{boxNote}
\begin{boxNote}{}
    Face latices can get complicated
    \begin{boxTODO}{}
        put sage output here.
    \end{boxTODO}
\end{boxNote}

\begin{boxNote}{}
    here are some combinatorial shapes
    \begin{itemize}
        \item simpliex
        \item (?)
        \item permutathedron
        \item associahedron
    \end{itemize}
\end{boxNote}


\begin{boxNote}{}
\begin{boxProblem}{}
    does every poset correspond to a face lattice.
    \tcblower
    Face lattices need bo be graded,
    even further, they need to be eulerian.
    \\
    It is an open question:
    "what are a list of requirements for a poset to be a face lattice"
\end{boxProblem}
\end{boxNote}

\section{The permutahedron}
\begin{boxConcept}{}
    \( R_d \subseteq \RR^d \) is the convex hull of the points obtained by permuting the coordinates of:
    \[ (1, 2, \cdots d ) \]
    \begin{boxExample}{}
        For example \( d = 3 \);
        \[
        \left\{
        \begin{array}{c}
            (1, 2, 3),\\
            (1, 3, 2),\\
            (2, 1, 3),\\
            (2, 3, 1),\\
            (3, 1, 2),\\
            (3, 2, 1)
        \end{array}
        \right\}
        \]
    \end{boxExample}
    \begin{boxTheorem}{}
        The dimension of \( R_d \) is \( (d-1) \).
    \end{boxTheorem}
    \begin{boxTODO}{} draw \end{boxTODO}
    \begin{boxNote}{}
        Edges connect vertices where a single transposition is preformed and the two values being tranposed differ by 1.
    \end{boxNote}
    Now, we will represent this is a different notation.
    \[ a|b|c \]
    where
    \begin{itemize}
        \item \( a,b,c \in \{1, 2, \cdot, d \} \)
        \item \( a \) is the first coordinate
        \item \( b \) is the second coordinate
        \item \( c \) is the third coordinate
    \end{itemize}
        \[
        \left\{
        \begin{array}{c}
            (1, 2, 3),\\
            (1, 3, 2),\\
            (2, 1, 3),\\
            (2, 3, 1),\\
            (3, 1, 2),\\
            (3, 2, 1)
        \end{array}
        \right\}
        =
        \left\{
        \begin{array}{c}
            1 | 2 | 3\\
            1 | 3 | 2\\
            2 | 1 | 3\\
            3 | 1 | 2\\
            2 | 3 | 1\\
            3 | 2 | 1
        \end{array}
        \right\}
        \]
\end{boxConcept}

Now, we stop for an interlude

\section{ordered set partitions}
\begin{boxTODO}{}
    
\end{boxTODO}

