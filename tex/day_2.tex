\chapter{Polyhedral Polyhedral Geometry}
\begin{boxConcept}{}
    In Geometry and Research, it is harder to figure out what is true rather than how to prove it.

    Having an intuitive picture of what's going on makes this a lot easier.
\end{boxConcept}


\section{Polyhedra}
\begin{boxDefinition}{Half Space}
    This is the solution set to an affine linear inequality:
    \begin{boxDefinition}{Affine Linear Inequality}
        An inequality in the terms of:
        \[ a_1x_1 + a_2x_2 + \cdots + a_dx_d \leq b \]
        Where \( a_i \in \RR \) (for us)
        \begin{boxDefinition}{Affine}
        Affine refers to \( b \neq 0 \). If \( b \) were strictly equal to 0, then this would be a strict linear inequality.
        \end{boxDefinition}
    \end{boxDefinition}
\end{boxDefinition}
\begin{boxDefinition}{Polyhedron}
    The intersection of finitely many half-spaces in \( \RR^d \).
    \begin{boxNote}{Polyhedron aren't nescissarily bounded.}
    \end{boxNote}
\end{boxDefinition}

\begin{boxExample}{Simplex in \texorpdfstring{\(\RR^2\)}{R2}}
    \begin{boxTODO}{}
        This is the convex hull of \( (0,0), (0,1), (1,0) \)
    \end{boxTODO}
    \begin{boxTODO}{}
    Fill in picture
    \end{boxTODO}
    \[
    \left\{
    \begin{array}{rl}
        x_1 \geq 0\\
        x_2 \geq 0\\
        x_1 + x_2 \leq 1\\
    \end{array}
    \right.
    \]
\end{boxExample}

\begin{boxExample}{Simplex in \texorpdfstring{\(\RR^3\)}{R3}}
    \begin{boxTODO}{}
        This is the convex hull of \( (0,0,0), (0,0,1), (0,1,0), (1,0,0) \)
    \end{boxTODO}
    \begin{boxTODO}{}
    Fill in picture
    \end{boxTODO}
    \[
    \left\{
    \begin{array}{rl}
        x_1 &\geq 0\\
        x_2 &\geq 0\\
        x_3 &\geq 0\\
        x_1 + x_2 + x_3 &\leq 1
    \end{array}
    \right.
    \]
\end{boxExample}
\begin{boxTODO}{Unbounded Polyhedron Example}
\end{boxTODO}
\begin{boxDefinition}{Simplex in \texorpdfstring{\(\RR^d\)}{Rd}}
    We gave the examples of
    \begin{boxTODO}{}
        Reference R1 and R2
    \end{boxTODO}
\end{boxDefinition}
\begin{boxDefinition}{Cube in \texorpdfstring{\(\RR^d\)}{Rd}}
    \begin{boxTODO}{}
        Insert cube def
    \end{boxTODO}
\end{boxDefinition}
\begin{boxProblem}{Octahedron Problem}
    \begin{boxTODO}{}
        Inser octahedron from lecture.
    \end{boxTODO}
\end{boxProblem}
\begin{boxDefinition}{Cross Polytope}
    \begin{boxTODO}{}
    Later
    \end{boxTODO}
\end{boxDefinition}

\begin{boxNote}{}
    Each facet has in inequality associcated with it.
\end{boxNote}
\begin{boxTheorem}{}
    There is a unique, irredundent, half spaces to describe any polyhedron.
    \begin{boxNote}{}
        Irrendundent = no extras
    \end{boxNote}
\end{boxTheorem}


\section{Bounded Polyhedra (Polytope)}
\begin{boxConcept}{Polytope}
    \begin{boxDefinition}{Polytope}
        A polytope is a bounded polyhdron.
        \begin{boxTODO}{}
        Define bounded
        \end{boxTODO}
    \end{boxDefinition}
    \begin{boxDefinition}{Convex Hull}
        The convex hull of a set of points, is given by:
        \[ \Conv\{v_1, v_2, \cdots, v_n\} = 
            \{
                \lambda_1v_1 + \cdots \lambda_nv_n
                \ :\ \
                lambda_i \in \RR^\geq 0
                ,\ 
                \lambda_1 + \cdots + \lambda_n = 1 
            \}
        \]
        \begin{boxNote}{Shrink Wrap}
            This is the unique smallest convex volume that contains all of \( v_i \).\\
            This is obtained by "shrink wrapping", in a non-rigerous sense, the points of \( v_i \).
        \end{boxNote}
    \end{boxDefinition}
    \begin{boxTheorem}{All Polytopes can be Expressed as a Convex Hull}
    \end{boxTheorem}
    \begin{boxExample}{}
    \begin{boxTODO}{}
        Give an example
    \end{boxTODO}
    \end{boxExample}
\end{boxConcept}


\begin{boxConcept}{V and H descriptions}
    \begin{boxDefinition}{\texorpdfstring{\(H\)}{H}-Description}
        The \( H \)-description of a polyhedra is in terms of \textbf{H}alfspaces.
    \end{boxDefinition}
    \begin{boxDefinition}{\texorpdfstring{\(V\)}{V}-Description}
        The \( V \)-description of a polytope is in terms of \textbf{V}ectors.\\
        That is, convex hulls.
        Note that only bounded polyhedra (polytopes) have a \( V \)-description.
    \end{boxDefinition}
\end{boxConcept}

\section{Cones}
\begin{boxDefinition}{Cone}
    A cone is a polyhedron where every half-space in its \( H \)-description has the origin on its boundary.
    \begin{boxTheorem}{Alternate Cone Definition}
        If each halfspace has 0 in its inequality. That is, \( b_i = 0 \) for all \( i \).
    \end{boxTheorem}
\end{boxDefinition}
\begin{boxProblem}{}
    All cones, except one, are unbounded.
    Which cone is unbounded?
\end{boxProblem}
\begin{boxExample}{Example of a cone}
    \begin{boxTODO}{}
        Cone
    \end{boxTODO}
\end{boxExample}



\section{Dimension of a Polyhedron}
\begin{boxExample}{Motivation for Dimension}
    Let
    \[ P := \Conv\{ (1, 2), (2,3) \} \subset \RR^2 \]
    \begin{boxTODO}{}
    Insert image
    \end{boxTODO}
    If we naively assign \( \Dim P = 2 \), because it sits in \( \RR^2 \), it wouldn't be much help, as \( P \) is in reality a line.
\end{boxExample}
\begin{boxDefinition}{Dimension of a Polyhedron}
    The dimension of a polyhedra is:
    \[ \Dim(P) := \Dim \Span \{ x-y\ :\ x,y\in P \} \]
    \tcblower
    This merits some explaination.\\
    We note if \( x = x \), then this gives us \( x - x = 0 \), which is simply the origin.
    \begin{boxTODO}{}
    Explain better
    \end{boxTODO}
\end{boxDefinition}

\section{Faces of a Polyhedron}
\begin{boxTODO}{Define Facet}
    Did we define facets?
\end{boxTODO}

\begin{boxDefinition}{Face}
    A Face of \( P \) is the intersection of \( P \) with the boundary of a halfspace containing \( P \).
\end{boxDefinition}
\begin{boxTODO}{}
    Insert exmaples
    \begin{boxExample}{}
    \end{boxExample}
    \begin{boxExample}{}
    \end{boxExample}
\end{boxTODO}
\begin{boxDefinition}{Facet}
    Let \( P \) be a polyhedron. A face \( F \) of \( P \) is a facet if:
    \[ \Dim(F) = \Dim(P) - 1 \]
\end{boxDefinition}
\begin{boxDefinition}{Edge}
    Let \( P \) be a polyhedron.
    A face \( F \) of \( P \) is an edge if
    \[ \Dim(F) = 1 \]
\end{boxDefinition}
\begin{boxTODO}{}
\begin{boxDefinition}{Ridge}
\end{boxDefinition}
\begin{boxDefinition}{Co-Dimension}
\end{boxDefinition}
\end{boxTODO}

\begin{boxTheorem}{}
    Every face of a polyhedron is itself a polyhedron.
\end{boxTheorem}


\begin{boxExample}{The 4-cube}
    Find all of the faces of the 4-cube.
    \tcblower
\end{boxExample}
