%
\chapter{Introduction}
\section{Properties of Numerical Semigroups}
\begin{boxDefinition}{Numerical Semigroup}
    \begin{boxNote}{}
        Zero is a generator of every numerical subgroup, it is just not explicitly mentioned.
    \end{boxNote}
\end{boxDefinition}
\begin{boxExample}{}
    Let \( S := \gen{6, 9, 20} \)
    Then:
    \[ S = \{ \} \]
    \begin{boxTODO}{}
        Fill in mcnugget set.
        Fix wording below.
    \end{boxTODO}
    \begin{boxNote}{}
        This is called the "McNugget Semigroup", as McDonald's Chicken Nuggets used to come in sizes 6, 9, or 20.
        The question was, "what's the largest number of McNuggets you cannot recieve from McDonalds"?
    \end{boxNote}
\end{boxExample}
\begin{boxTheorem}{}
    Every numerical subgroup has a unique minimal generating set.
    \begin{boxExample}{}
        \[ S = \gen{6, 9, 20} = \gen{6, 9, 18, 20} \]
    \end{boxExample}
\end{boxTheorem}
\begin{boxDefinition}{Multiplicity}
    Let \( S \) be a numerical subgroup.
    The multiplicity of \( S \), denoted \( m(S) \), is the minimal nonzero generator of \( S \).
    That is:
    \[ m(S) := \Min(S \setminus \{ 0 \} ) \]
\end{boxDefinition}
\begin{boxDefinition}{Embedding Dimension}
    Let \( S \) be a numerical subgroup.
    The embedding dimension of \( S \), denoted \( e(S) \), is the number of minimal nonzero generators of \( S \).
    \begin{boxExample}{}
        \[ e(\gen{6, 9, 20} = 3 \]
    \end{boxExample}
\end{boxDefinition}
\begin{boxDefinition}{Frobenius Number}
    Let \( S \) be a numerical subgroup.
    The Frobenius Number of \( S \), denoted \( F(S) \), is the largest non-negative integer not in \( S \).
    That is:
    \[ F(S) := \Max(\ZZ_{\geq 0} \setminus S) \]
\end{boxDefinition}
\begin{boxNote}{}
    Often we will require:
    \[ \abs{\NN \setminus S} < \infty \]
    So that these properties are well defined.
\end{boxNote}
\begin{boxDefinition}{}
    
\end{boxDefinition}
\begin{boxExample}{The Four Properties}
    Let \( S := \gen{6, 9, 20} \). Then:
    \begin{align*}
        m(S) &= 6 &&  \text{Smallest generator}\\
        e(S) &= 3 &&  \text{Number of generators}\\
        F(S) &= 43 && \text{Largest "gap"}\\
        g(S) &= 22 && \text{Number of "gaps"}
    \end{align*}
\end{boxExample}


\begin{boxConcept}{}
    \begin{boxDefinition}{}
        Semigroup isomorphism
        \begin{boxTODO}{}
            ?
        \end{boxTODO}
    \end{boxDefinition}
    \begin{boxExample}{}
        \begin{align*}
            S &:= \gen{4, 6}\\
            T &:= \gen{2, 3}
        \end{align*}
        We claim, without proving:
        \[ S \isoto T \]
        We notice \( \NN \setminus S \) is infinite, but \( \NN \setminus T \) is not.
    \end{boxExample}
\end{boxConcept}


\begin{boxConcept}{Factorizations}
    If \( S := \gen{6, 9, 20} \) then:
    \[ 60 = 4\cdot 6 + 4\cdot 9 = 3\cdot 20 \]
    This shows factorizations are not unique.
    So we define the set of all vectors:
    \begin{boxDefinition}{}
        \[ Z_S(\ell) := \{ a \in \NN^k\ :\ a_1n_1 + \cdots a_kn_k = \ell \} \]
        Where \( \ell \in S \).
    \end{boxDefinition}
    \begin{boxExample}{}
        \[
            Z(60) = 
            \left\{
            \begin{array}{c}
            (10, 0, 0),\\
            (7, 2, 0),\\
            (4, 4, 0),\\
            (0, 0, 3),\\
            (1, 6, 0)
            \end{array}
            \right\}
        \]
    \end{boxExample}
\end{boxConcept}

\section{The Apery Set}
\begin{boxDefinition}{}
    Let \( S \subseteq \NN \) be a numerical semigroup.\\
    Let \( m := m(S) \) be the multiplicity of \( S \).\\
    Then we define the Apery set:
    \[ \Ap(S) := \{ n\in S\ :\ n-m \not\in S \]
    \begin{boxNote}{}
        These can be though of as the element "just barely inside" of \( S \)
    \end{boxNote}
    \begin{boxNote}{}
        Note the elements of the Apery set are distinct modulo \( m \).
    \end{boxNote}
\end{boxDefinition}
\begin{boxProblem}{}
    Let:
    \[ S := \gen{3, 5, 7} \]
    \tcblower
    This implies:
    \[ S = \{ 0, 3, \underbrace{5, 6, 7}, 8, 9, 10, \cdots \} \]
    We want the smallest element in each equivalence class mod \( 3 \), which is:
    \[ \Ap(S) = \{ 0, 7, 5 \} \]
    %\[ S = \{ 0, 3, \underbrace{5, 6, 7}, 8, 9, 10, \cdots \} \]
    %Since 3 is the smallest generator, and 3 consecutive integers are included in \( S \), 
\end{boxProblem}
\begin{boxNote}{}
    We note the following properties of \( \Ap(S) \):
    \begin{itemize}
        \item \( \abs{\Ap(S)} = m \)
        \item the minimal generators of \( S \) are (except \( m (S) \) are in \( \Ap(S) \).
    \end{itemize}
\end{boxNote}
% around 1:00:00 into 2021-06-28
\begin{boxDefinition}{Apery Poset of S}
    \begin{boxTODO}{}
        
    \end{boxTODO}
\end{boxDefinition}
